\documentclass{beamer}

% \input{LatexTemplates/Template.tex}

\usepackage{graphicx,xspace}
%% \usepackage{dougplorm}
\usepackage[normalem]{ulem}
\usepackage{grffile}
\usepackage{hyperref}
\usepackage{url}

\begin{document}


\begin{frame}


\frametitle{Course introduction}\tableofcontents[hideallsubsections]
\end{frame}


 \section{Introduction }

 \subsection{Ground rules}
\begin{frame}


\frametitle{Expectations of professor}\begin{itemize}

\item Start and end on time

\item Focus on conceptual understanding

\item Make clear what terminology and facts must be learned

\item Open to questions -- both in class (within reason) and at office
	hours

\item Available by email and on Facebook\begin{itemize}

\item I am not your friend

\item At least, not on Facebook\end{itemize}\end{itemize}
\end{frame}

\begin{frame}


\frametitle{Expectations of students}\begin{itemize}

\item Start and end on time

\item Print the notes from the web and bring them to class

\item Don't talk while other students are talking, or while I am
	responding to student questions

\item If you must talk at other times, be inobtrusive

\item Don't use your computer in class

\item If you must use your computer in class, be inobtrusive\begin{itemize}

\item And don't connect to the internet\end{itemize}\end{itemize}
\end{frame}

\begin{frame}


\frametitle{Structure of presentation}\begin{itemize}

\item Required material will be clearly outlined in the notes\begin{itemize}

\item * {\color{blue} This is an answer: it was omitted from the notes for
		discussion purposes, you should probably write it in}

\item {\color{green}\sl This is a comment: I omitted from the notes because I
		thought it wasn't necessary for you to study}\end{itemize}

\item Required terminology will be presented in {\bf bold}

\item General ideas and approaches presented in class may also be required;
	you should take notes on these in your own words\end{itemize}
\end{frame}

\begin{frame}


\frametitle{Why come to class?}\begin{itemize}

\item It's required

\item Listening and thinking and talking will help you understand
	concepts, instead of just memorizing

\item Details and terminology should be covered in sufficient detail in
	the notes; concepts may not be

\item {\color{green}\sl You can't get your money back, so you may as well enjoy
	the show}\end{itemize}
\end{frame}

\begin{frame}


\frametitle{Why read the book?}\begin{itemize}

\item It's interesting

\item The book will explain some things in a better way (for you
	personally) than I do

\item Familiarity improves understanding\end{itemize}
\end{frame}

\begin{frame}


\frametitle{Taking notes}\begin{itemize}

\item You will need to develop your own style of taking notes\begin{itemize}

\item Many people benefit from writing things down, or using their own words\end{itemize}

\item If a new concept is making sense to you right now, write something
	that will help you remember

\item If there's something I think you all need to write down, I will write
	it for you (or mark it as an answer)\end{itemize}
\end{frame}

\begin{frame}


\frametitle{Evaluation}\begin{itemize}

\item You are not responsible for details unless they are in the notes\begin{itemize}

\item and not responsible for terminology unless it's in {\bf bold}\end{itemize}

\item You {\em are\,} responsible for relevant ideas and concepts from lectures
	and readings

\item Conceptual questions, logical inference questions and application questions
	are fair game\begin{itemize}

\item Practice questions will be available\end{itemize}\end{itemize}
\end{frame}

\begin{frame}


\frametitle{{\sl Structure of presentation}}\begin{itemize}

\item Required material will be clearly outlined in the notes\begin{itemize}

\item * {\color{blue} This is an answer: it was omitted from the notes for
		discussion purposes, you should probably write it in}

\item {\color{green}\sl This is a comment: I omitted from the notes because I
		thought it wasn't necessary for you to study}\end{itemize}

\item Required terminology will be presented in {\bf bold}

\item General ideas and approaches presented in class may also be required;
	you should take notes on these in your own words\end{itemize}
\end{frame}


 \section{Thinking conceptually}
\begin{frame}


\frametitle{Deductive thinking}\begin{itemize}

\item Science proceeds by advancing hypotheses and comparing them to facts

\item Facts can be observed from nature, or we can construct experiments
	to test specific hypotheses

\item Basic, logical thinking is very {\em simple,\,} but it is often not {\em easy\,} for
	humans to think clearly about abstract concepts\begin{itemize}

\item {\color{green}\sl Which is more complicated: algebra or hockey?}\end{itemize}\end{itemize}
\end{frame}

\begin{frame}


\frametitle{{Algebra}}

\hfill\includegraphics[width=1\textwidth]{webpix/quadratic.jpg}\hfill\mbox{}
\end{frame}

\begin{frame}


\frametitle{{Hockey}}

\hfill\includegraphics[height=0.8\textheight]{webpix/hockey.jpg}\hfill\mbox{}
\end{frame}


 \subsection{Example: cards and drinks}
\begin{frame}


\frametitle{{Cards}}

\hfill\includegraphics[height=0.8\textheight]{webpix/cards.jpg}\hfill\mbox{}
\end{frame}

\begin{frame}


\frametitle{{\sl Deductive thinking}}\begin{itemize}

\item You go to a job interview, and are shown some playing cards.

\item Some cards are face up, and you can see that they are aces or kings.

\item Some cards are face down, and you can see whether they have bicycles
	or airplanes on the back

\item The interviewer asks you to test the hypothesis that all of the aces
	have airplanes on the back

\item Which of the four groups of cards do you need to turn over?\begin{itemize}

\item * {\color{blue} }\end{itemize}\end{itemize}
\end{frame}

\begin{frame}


\frametitle{{Drinks}}

\hfill\includegraphics[height=0.8\textheight]{webpix/bar.jpg}\hfill\mbox{}
\end{frame}

\begin{frame}


\frametitle{Deductive thinking}\begin{itemize}

\item You are the manager of a restaurant

\item You can see some people's drinks clearly, and tell whether the
	drinks are alcoholic or not (but not the people's ages)

\item You can see other people's faces clearly, and tell whether they are
	underage or legal-age (but not what they are drinking)

\item You want to test the hypothesis that everything is OK:\begin{itemize}

\item everybody who is drinking alcohol is of legal age\end{itemize}

\item Which of the four groups of people do you need to check out?\begin{itemize}

\item * {\color{blue} The underage people, and the alcohol drinkers}\end{itemize}\end{itemize}
\end{frame}

\begin{frame}


\frametitle{Deductive thinking}\begin{itemize}

\item You go to a job interview, and are shown some playing cards.

\item Some cards are face up, and you can see that they are aces or kings.

\item Some cards are face down, and you can see whether they have bicycles
	or airplanes on the back

\item The interviewer asks you to test the hypothesis that all of the aces
	have airplanes on the back

\item Which of the four groups of cards do you need to turn over?\begin{itemize}

\item * {\color{blue} The aces and the cards with bicycles on the back}\end{itemize}\end{itemize}
\end{frame}

\begin{frame}


\frametitle{Thinking conceptually}\begin{itemize}

\item Logical interpretation and inference is simple, but not always easy\begin{itemize}

\item This is true for everyone\end{itemize}

\item Being on familiar ground helps us think clearly\begin{itemize}

\item This will work for different people in different ways: learning
		facts, stories, mechanisms, etc.\end{itemize}

\item Practice clear thinking about simple questions\end{itemize}
\end{frame}


 \subsection{Logical inference}
\begin{frame}


\frametitle{Inference}\begin{itemize}

\item Does the last statement {\em follow\,} from the first two?

\item Cats have four legs.  Mammals have four legs.  {\em Therefore\,}, cats are mammals\begin{itemize}

\item * {\color{blue} Not a valid conclusion}\end{itemize}

\item Cows can fly.  Dushoff is a cow.  {\em Therefore\,}, Dushoff can fly.\begin{itemize}

\item * {\color{blue} Valid conclusion}

\item * {\color{blue} Based on the assumptions}\end{itemize}\end{itemize}
\end{frame}


\end{document}

