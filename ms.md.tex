\subsection{Abstract}\label{abstract}

\section{Introduction}\label{introduction}

\section{Background and problem
setup}\label{background-and-problem-setup}

R and r. R is often thought of as more important; r is easier to measure
earlier in an epidemic. They are linked through the generation interval,
and this is usually done using the generating function approach
popularized by \autocite{WallLips07}.

Much infectious disease modeling focuses on estimating the reproductive
number -- the number of new cases caused on average by each case. In the
specific case where the case is introduced in a fully susceptible
population, we talk about the basic reproductive number \Ro. The
reproductive number provides information about the disease's potential
for spread and the difficulty of control. It is often thought of a
single number: an average \autocite{AndeMay91} or an appropriate sort of
weighted average \autocite{DiekHees90}. But the reproductive number can
also be thought of as a distribution across the population of possible
infectors: different hosts may have different tendencies to transmit
disease.

The reproductive number provides information about how a disease
spreads, on the scale of disease generations. It does not, however,
contain information about the population-level rate of spread (e.g.~how
disease incidence increases through time, which can be critical for
public health interventions). Hence, another important quantity is the
population-level \emph{rate of spread}. In disease outbreaks, the rate
of spread is often inferred from case-incidence reports and used to
estimate the reproductive number.

\(i(t) \approx i(0) \exp(rt)\)

\(T_c = 1/r\)

\(T_2 = \ln(2)/r\)

\(r_0\) can be observed early in the epidemic

\(r\) can typically be measured more robustly than \(\Reff\)

The reproductive number and the rate of spread are linked by the
\emph{generation interval} -- the interval between the time that an
individual is infected by an infector, and the time that the infector
was infected \cite{Sven07}.

Whereas the rate of spread measures the speed of the disease at the
population level, the generation interval measures speed at the
individual level. It is typically inferred from contact tracing,
sometimes in combination with clinical data. Like the reproductive
number, the generation interval can be thought of as a single number
(typically its mean), or as a distribution.

Here, we extend the work of \autocite{WallLips07} in two ways: we
re-interpret their ``generating-equation approach'' to calculating R as
a ``filtered mean'', and discuss the properties and interpretations of
filtered means; and we suggest an alternative, more tractable moment
approximation for the relationship between r and R.

\subsection{Overview}\label{overview}

We are interested in the relationship between \(r\), \Rx~and the
generation-interval distribution.

We define the generation-interval distribution using a renewal-equation
approach. A wide range of disease models can be described using this
model: \[i(t) = S(t)\int{K(s)i(t-s) \,ds},\] where \(t\) is time,
\(i(t)\) is the incidence of new infections, \(S(t)\) is the
\emph{proportion} of the population susceptible, and \(K(s)\) is the
intrinsic infectiousness of individuals who have been infected for a
length of time \(s\).

We then have the basic reproductive number: \[\Ro = \int{K(s)ds},\] and
the \emph{intrinsic} generation-interval distribution:
\[g(s) = \frac{K(s)}{\Ro}\] (the ``intrinsic'' interval can be
distinguished from ``realized'' intervals, which can look ``forward'' or
``backward'' in time \autocite{ChamDush15}, see also earlier work {[}@

\begin{verbatim}
Where:

    \Rx\ is the effective reproductive number

    $g(\tau)$ (integrates to 1)  
    is the \emph{intrinsic} generation distribution
\end{verbatim}

\begin{center}\rule{0.5\linewidth}{\linethickness}\end{center}

Euler equation

\begin{verbatim}
Model

    $$i(t) = \Rx\int{g(\tau)i(t-\tau) \,d\tau}$$

Exponential phase

    $$i(t) = i(0) \exp(t/C)$$

Conclusion

    $$1/\Rx = \int{g(\tau)\exp(-\tau/C) \,d\tau}$$
\end{verbatim}

\begin{center}\rule{0.5\linewidth}{\linethickness}\end{center}

Interpretation: the ``effective'' generation time

\begin{verbatim}
If the generation interval were absolutely fixed at a time interval
of $G$, then 

    $${\Rx} = \exp(G/C)$$

\emph{Define} the effective generation time so that this remains
true:

    $${\Rx} = \exp(\hat G/C)$$
\end{verbatim}

\begin{center}\rule{0.5\linewidth}{\linethickness}\end{center}

A filtered mean

\begin{verbatim}
If:

    $${\Rx} = \exp(\hat G/C)$$

Then

    $$1/\Rx = \int{g(\tau)\exp(-\tau/C) \,d\tau}$$

Becomes

    $$\exp(- \hat G/C) = \int{g(\tau)\exp(-\tau/C) \,d\tau}$$
    
    or, $$\exp(-\hat G/C) = \langle \exp(-\tau/C) \rangle_g$$,

This is a ``filtered mean" of the distribution $g$.

Equivalent to the Wallinga and Lipsitch generating function
\end{verbatim}

\begin{center}\rule{0.5\linewidth}{\linethickness}\end{center}

Filtered means

\begin{verbatim}
Many things we know about are examples of filtered means

    Geometric mean (log function)

    Harmonic mean (reciprocal function)

    Root mean square (square) 

    Heterogeneous \Rx\ calculations
\end{verbatim}

Derivation

Examples

Drawbacks

\subsection{Moment approximation}\label{moment-approximation}

Derivation

Examples

\subsection{Discussion}\label{discussion}

\subsection{Acknowledgments}\label{acknowledgments}

\printbibliography
